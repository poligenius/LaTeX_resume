%%%%%%%%%%%%%%%%%
% This is an example CV created using altacv.cls (v1.1, 21 November 2016) written by
% LianTze Lim (liantze@gmail.com), based on the 
% Cv created by BusinessInsider at http://www.businessinsider.my/a-sample-resume-for-marissa-mayer-2016-7/?r=US&IR=T
% 
%% It may be distributed and/or modified under the
%% conditions of the LaTeX Project Public License, either version 1.3
%% of this license or (at your option) any later version.
%% The latest version of this license is in
%%    http://www.latex-project.org/lppl.txt
%% and version 1.3 or later is part of all distributions of LaTeX
%% version 2003/12/01 or later.
%%%%%%%%%%%%%%%%

%% If you want to use \orcid or the
%% academicons icons, add "academicons"
%% to the \documentclass options. 
%% Then compile with XeLaTeX or LuaLaTeX.
% \documentclass[10pt,a4paper,academicons]{altacv}
\documentclass[10pt,a4paper]{altacv}

%% AltaCV uses the fontawesome and academicon fonts
%% and packages. 
%% See texdoc.net/pkg/fontawecome and http://texdoc.net/pkg/academicons for full list of symbols.
%% When using the "academicons" option,
%% Compile with LuaLaTeX for best results. If you
%% want to use XeLaTeX, you may need to install
%% Academicons.ttf in your operating system's font %% folder.


% Change the page layout if you need to
\geometry{left=1cm,right=9cm,marginparwidth=6.8cm,marginparsep=1.2cm,top=1cm,bottom=1cm}

% Change the font if you want to.

% If using pdflatex:
\usepackage[utf8]{inputenc}
\usepackage[T1]{fontenc}
\usepackage[default]{lato}
\usepackage{setspace}
\usepackage{hyperref}
\hypersetup{colorlinks=true, urlcolor=blue}

% If using xelatex or lualatex:
% \setmainfont{Lato}

% Change the colours if you want to
\definecolor{DarkBlue}{HTML}{0b2b5f}
\definecolor{Black}{HTML}{000000}

\colorlet{heading}{DarkBlue}
\colorlet{accent}{DarkBlue}
\colorlet{emphasis}{Black}
\colorlet{body}{Black}

% Change the bullets for itemize and rating marker
% for \cvskill if you want to
\renewcommand{\itemmarker}{{\small\textbullet}}
\renewcommand{\ratingmarker}{\faCircle}


%% sample.bib contains your publications
\addbibresource{sample.bib}

\begin{document}
\name{Marco Marini}
  \tagline{Computer Science Engineer}
% Cropped to square from https://en.wikipedia.org/wiki/Marissa_Mayer#/media/File:Marissa_Mayer_May_2014_(cropped).jpg, CC-BY 2.0
% foto circolare
\photo{4cm}{sagoma}



\personalinfo{%
  % Not all of these are required!
  % You can add your own with \printinfo{symbol}{detail}
  \doublespacing
  \email{\href{mailto:Marco.97Marini@gmail.com}{Marco.97Marini@gmail.com}}
  \linkedin{\href{https://www.linkedin.com/in/marco-marini-0662351a2}{Marco Marini}}
  \github{\href{https://github.com/poligenius?tab=repositories}{poligenius}}
  \homepage{\href{https://mmarini.it}{personal website}}
  \phone{+39 3336582733}
  % \present{22/10/1997}
  \location{Italy, Milan}
  % \car{Class B}
  \salary{44.000 €}
% If you want to use this field (and also other academicons symbols), add "academicons" option to \documentclass{altacv}
}



%% Make the header extend all the way to the right, if you want. Extend the right margin by 8cm (=6.8cm marginparwidth + 1.2cm marginparsep)
\begin{adjustwidth}{}{-8cm}
\makecvheader
\end{adjustwidth}

%% Provide the file name containing the sidebar contents as an optional parameter to \cvsection.
%% You can always just use \marginpar{...} if you do
%% not need to align the top of the contents to any
%% \cvsection title in the "main" bar.
\cvsection[page1sidebar]{About Me}
\textbf{AI}, \textbf{data}, and \textbf{software engineering} professional with experience in developing innovative tech solutions by coordinating cross-functional teams. 
\\For a more comprehensive overview of my background and personal projects, please visit my \href{https://mmarini.it}{personal website}.

%\begin{document}
%    \begin{figure}
%        \centering
%            \includegraphics[width=0.30\textwidth]{Exito1.jpg}
%            \centering
%    \end{figure}

% \divider

% \cvevent{Product Engineer}{Google}{23 June 1999 -- 2001}{Palo Alto, CA}

% \begin{itemize}
% \item Joined the company as employe \#20 and female employee \#1
% \item Developed targeted advertisement in order to use user's search queries and show them related ads
% \end{itemize}

\cvsection{WORK EXPERIENCE}

\cvevent{\textbf{Tech Product Owner}}{\href{https://mediasetitalia.com}{Mediaset}}{Dec 2023 - Present}{Cologno Monzese (MI)}
     \textbf{Lead cross-functional teams} and coordinate with external providers (developers, designers, architects) to develop cutting-edge technological solutions that drive significant ROI.
     \begin{itemize}
        \item \textbf{Private Mobile Application}: Managed the \textbf{€500,000} end-to-end development of a private mobile app deployed on-premise. Led a team of 8 through the entire product lifecycle from design to delivery. The app is now used daily by around \textbf{300} Mediaset sales representatives. Backend was developed using \textbf{Typescript} and SQL for queries, while frontend with \textbf{React} and CSS.
        \item \textbf{RAG System}: Designed and implemented a system on Google Cloud Platform (\textbf{GCP}) using Gemini 1.5, enabling employees to query and access PowerPoint presentations, PDFs, and other marketing documents.
        \item \textbf{Recommendation Engine}: Led the development of a recommendation system deployed on \textbf{AWS} within Mediaset’s cloud. Part of a larger \textbf{€3 million} project, this system assists operations teams in assigning advertisement spots based on customer requests, with an \textbf{estimated annual ROI} of \textbf{€5 million}. Backend and models were developed using Python (\textbf{PySpark} for backend integrations and \textbf{Scikit Learn + Tensorflow} for models).
        \item \textbf{Generative AI Evangelist:} Spearheaded an internal analysis to identify opportunities for integrating Generative AI SaaS or custom solutions, aiming to achieve significant savings in full-time equivalents (FTEs) and enhance operational efficiency.
     \end{itemize}


\cvevent{\textbf{Associate - Data \& Analytics - AI \& ML Engineer}}{\href{https://www.pwc.com/it/it.html}{PwC Italy}}{Oct 2022 - Dec 2023}{Milan, Citylife}
    \begin{itemize}
        \item Developed and maintained the data pipeline for one of the largest Italian financial institutions, reducing build time by \textbf{50\%} and significantly decreasing both computational costs and data storage requirements. Worked on Palantir, utilizing \textbf{PySpark} and \textbf{SQL}.
    \end{itemize}
    \begin{itemize}
        \item Developed a complex application for calculating risk indicators used by auditors, implementing the \textbf{backend} using \textbf{PySpark} and the \textbf{frontend} with \textbf{React}. \textbf{Integrated forecasting models} to anticipate market trends and enable proactive or corrective actions.
    \end{itemize}

%\cvsection{Top 5 Soft Skills}
%\cvtag{Problem Solving}
%\cvtag{Creativity}
%\cvtag{Team Work}
%\cvtag{Fast Learning}
%\cvtag{Listening}

%\clearpage

%\begin{document}
%    \begin{figure}
%        \centering
%            \includegraphics[width=0.30\textwidth]{Exito1.jpg}
%            \centering
%    \end{figure}

%\end{document}

\end{document}
